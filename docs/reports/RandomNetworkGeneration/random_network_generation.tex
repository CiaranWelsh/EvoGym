\documentclass[../Thesis.tex]{subfiles}





\begin{document}
    \chapter{Generating Random Networks}
%    \begin{centering}
%        \abstractname
%    \end{centering}

    \section{Purpose}
    Why do we need random networks?

    We plan on using random networks as a basis for evolution of models

    \section{Requirements}
    What are the requirements for a random network generator?

    Properties such as steady state?

    \section{Implementation}
    The current implementation of RandomNetworkGenerator requires a single argument,
    and thats an instance of a RandomNetworkGeneratorOptions object, which holds
    all the options. These options include
    \begin{itemize}
        \item nCompartments: The number of compartments. Initialized from contineous uniform distribution
        between "compartmentLowerBound" and "compartmentUpperBound"
        \item nFloatingSpecies.
        These are initialized with a random variable draw from continueous uniform
        distribution between "speciesLowerBound and "speciesUpperBound".
        \item nBoundarySpecies.
        These are initialized with a random variable drawn from a discrete uniform distribution between
        "boundarySpeciesLowerBound" and "boundarySpeciesUpperBound"
        \item rateLaws.
        Define the rate laws that are randomly selected during network generation.
        See below for more details.
        \item motifs. Not yet implemented. The idea is a placeholder for randomly selecting entire motifs or sets of reactions with a predefined structure.
        \item seed. For achieving deterministic results.
    \end{itemize}

    The rate law argument requires a bit more detail.



    First a Roadrunner instance is created. This can optionally be seeded with an existing SBML model.
    Then compartments are created if there are more than

    \section{Problems}




\end{document}